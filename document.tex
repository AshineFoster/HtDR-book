\documentclass[11pt,a4paper]{report}
\usepackage{minted}
\usepackage{caption}
\usepackage{tabularx}
\usepackage{tabu}
\usepackage{longtable}
\usepackage{fancyvrb}
\usepackage[bookmarks,colorlinks]{hyperref}
\usepackage{bookmark}
\hypersetup{colorlinks,%
	citecolor=blue,%
	filecolor=blue,%
	linkcolor=blue,%
	urlcolor=blue,%
	}
\title{Design Recipes}
\begin{document}
	\maketitle
	\hypertarget{tocpage}{}
	\tableofcontents
	\bookmark[dest=tocpage,level=1]{Contents}
	\clearpage
	\hypertarget{tablist}{}
	\listoftables
	\bookmark[dest=tablist,level=1]{List of Tables}
	\setminted[racket]{autogobble, linenos, breaklines}
	\chapter{Design Recipes}
	In this course, we teach an approach to program design based on design recipes. Each recipe is
	applicable to certain problems, and systematizes the process of designing solutions to those
	problems.
		
	There are three core recipes that are used most frequently. The templating recipes are used as
	part of the design of every data definition and function. Abstraction recipes are used to reduce
	redundancy in code.
	\begin{table}[!h]
		\renewcommand{\arraystretch}{1.5}
		\renewcommand{\tabcolsep}{0.2cm}
		\begin{tabularx}{\textwidth}{| X | X | X | X |}
			\hline
			\textbf{Core Recipes}                                                                                              &                                                 \multicolumn{2}{c|}{\textbf{Templating}}                                                 & \textbf{Abstraction}                                                           \\
			\hline
			                                                                                                                   & \textbf{Data Driven}                                                                                       & \textbf{Control Driven}     &                                                                                \\
			\hline
			\emph{\nameref{ch:htdf}} Design any function.                                                         & \emph{\nameref{ch:data_driv_temp}} Produce template for a data definition based on the form of the type comment. & \emph{\nameref{sec:fun_comp}} & \emph{\nameref{sec:abs_from_eg}} Produce an abstract function given two similar functions. \\
			\hline
			\emph{\nameref{ch:htdd}} Produce data definitions based on structure of the information to be represented. & \emph{\nameref{ch:fun_2_1_of_data}} functions where 2 arguments have a one-of in their type comments.                     & \emph{\nameref{sec:back_srch}}  & \emph{\nameref{sec:abs_from_type_comm}} Produce a fold function given type comments.         \\
			\hline
			\emph{\nameref{ch:htdw}} Produce interactive programs that use big-bang.                                 &                                                                                                            & \emph{\nameref{sec:gen_recur}} &                                                                                \\
			\hline
			                                                                                                                   &                                                                                                            & \emph{\nameref{ch:accumulators}}         & \emph{\nameref{sec:using_abs_func}}                                                \\
			\hline
			                                                                                                                   &                                              \multicolumn{2}{c|}{\emph{\nameref{ch:temp_blend}}}                                               &                                                                                \\
			\hline
		\end{tabularx}
		\caption{Types of Design Recipes}
	\end{table}

	\chapter{How To Design Functions (HtDF)} \label{ch:htdf}
	The How to Design Functions (HtDF) recipe is a \textbf{design method} that enables systematic design of functions.
	 We will use this recipe throughout the term, although we will enhance it as we go to solve more complex problems.
	\\ \\
	\fbox{
	\parbox{12cm}{
	
	The HtDF recipe consists of the following steps:
	\begin{enumerate}
		\item Signature, purpose and stub.
		\item Define examples, wrap each in check-expect.
		\item Template and inventory.
		\item Code the function body.
		\item Test and debug until correct
	\end{enumerate}
	}}
	\\ \\
	NOTE:
	\begin{itemize}
		\item Each of these steps build on the ones that precede it. The signature helps write the purpose, the stub, and
		the check-expects; it also helps code the body. The purpose helps write the check-expects and code the
		body. The stub helps to write the check-expects. The check-expects help to code the body as well as to
		test the complete design.
		
		\item It is sometimes helpful to do the steps in a different order. Sometimes it is easier to write examples first,
		then do signature and purpose. Often at some point during the design you may discover an issue or
		boundary condition you did not anticipate, at that point go back and update the purpose and examples
		accordingly. But you should never write the function definition first and then go back and do the other
		recipe elements -- for some of you that will work for simple functions, but you will not be able to do that for
		the complex functions later in the course!
		
		\item Throughout the HtDF process be sure to "run early and run often". Run your program whenever it is well-
		formed. The more often you press run the sooner you can find mistakes. Finding mistakes one at a time is
		much easier than waiting until later when the mistakes can compound and be more confusing. Run, run, run!
	\end{itemize}

	\section{Signature, purpose and stub.}
	Write the function signature, a one-line purpose statement and a function stub.
	
	A signature has the type of each argument, separated by spaces, followed by ->, followed by the
	type of result. So a function that consumes an image and produces a number would have the
	signature Image -> Number.
	
	Note that the stub is a syntactically complete function definition that produces a value of the right
	type. If the type is Number it is common to use 0, if the type is String it is common to use "a" and
	so on. The value will not, in general, match the purpose statement. In the example below the stub
	produces 0, which is a Number, but only matches the purpose when double happens to be called
	with 0.
	
	\begin{minted}{racket}
	;; Number -> Number
	;; produces n times 2
	(define (double n) 0) ; this is the stub
	\end{minted}
	
	The purpose of the stub is to serve as a kind of scaffolding to make it possible to run the
	examples even before the function design is complete. With the stub in place check-expects that
	call the function can run. Most of them will fail of course, but the fact that they can run at all
	allows you to ensure that they are at least well-formed: parentheses are balanced, function calls 
	have the proper number of arguments, function and constant names are correct and so on. This is 
	very important, the sooner you find a mistake -- even a simple one -- the easier it is to fix.
	
	\section{Define examples, wrap each one in check-expect.}
	Write at least one example of a call to the function and the expected result the call should
	produce.
	
	You will often need more examples, to help you better understand the function or to properly test
	the function. If once your function works and you run the program some of the code is
	highlighted in black it means you definitely do not have enough examples. If you are unsure how
	to start writing examples use the combination of the function signature and the data definition(s)
	to help you generate examples. Often the example data from the data definition is useful, but it
	does not necessarily cover all the important cases for a particular function.
	
	The first role of an example is to help you understand what the function is supposed to do. If
	there are boundary conditions be sure to include an example of them. If there are different
	behaviours the function should have, include an example of each. Since they are examples first,
	you could write them in this form:
	
	\begin{minted}{racket}
	;; (double 0) should produce 0
	;; (double 1) should produce 2
	;; (double 2) should produce 4
	\end{minted}
	
	When you write examples it is sometimes helpful to write not just the expected result, but also
	how it is computed. For example, you might write the following instead of the above:
	
	\begin{minted}{racket}
	;; (double 0) should produce (* 0 2)
	;; (double 1) should produce (* 1 2)
	;; (double 2) should produce (* 2 2)
	\end{minted}
	
	While the above form satisfies our need for examples, DrRacket gives us a better way to write
	them, by enclosing them in check-expect. This will allow DrRacket to check them automatically
	when the function is complete. (In technical terms it will turn the examples into unit tests.)
	
	\begin{minted}{racket}
	;; Number -> Number
	;; produces n times 2
	(check-expect (double 0) (* 0 2))
	(check-expect (double 1) (* 1 2))
	(check-expect (double 3) (* 3 2))
	
	(define (double n) 0) ; this is the stub
	\end{minted}
	
	The existence of the stub will allow you to run the tests. Most (or even all) of the tests will fail
	since the stub is returning the same value every time. But you will at least be able to check that
	the parentheses are balanced, that you have not misspelled function names etc.
	
	\section{Template and inventory}
	Before coding the function body it is helpful to have a clear sense of what the function has to
	work with -- what is the contents of your bag of parts for coding this function? The template
	provides this.
	
	Once the How to Design Data Definitions (HtDD) recipe in introduced, templates are produced by
	following the rules on the Data Driven Templates section. You should copy the template from
	the data definition to the function design, rename the template, and write a comment that says
	where the template was copied from. Note that the template is copied from the data definition for
	the consumed type, not the produced type.
	
	For primitive data like numbers, strings and images the body of the template is simply ($\ldots$ x)
	where x is the name of the parameter to the function.
	
	Once the template is done the stub should be commented out.
	
	\begin{minted}{racket}
	;; Number -> Number
	;; produces n times 2
	(check-expect (double 0) (* 0 2))
	(check-expect (double 1) (* 1 2))
	(check-expect (double 3) (* 3 2))
	
	;(define (double n) 0) ; this is the stub
	
	(define (double n)     ; this is the template
		(... n))
	\end{minted}
	
	It is also often useful to add constant values which are extremely likely to be useful to the
	template body at this point. For example, the template for a function that renders the state of a
	world program might have an MTS constant added to its body. This causes the template to
	include an inventory of useful constants.
	
	\section{Code the function body}
	Now complete the function body by filling in the template.

	Note that:
	
	\begin{itemize}
		\item the signature tells you the type of the parameter(s) and the type of the data the function body
		must produce
		\item the purpose describes what the function body must produce in English
		\item the examples provide several concrete examples of what the function body must produce
		\item the template tells you the raw material you have to work with
	\end{itemize}

	You should use all of the above to help you code the function body. In some cases further
	rewriting of examples might make it more clear how you computed certain values, and that may
	make it easier to code the function.
	
	\begin{minted}{racket}
	;; Number -> Number
	;; produces n times 2
	(check-expect (double 0) (* 0 2))
	(check-expect (double 1) (* 1 2))
	(check-expect (double 3) (* 3 2))
	
	;(define (double n) 0) ; this is the stub
	
	;(define (double n)    ; this is the template
	; 	(... n))
	
	(define (double n)
		(* n 2))
	\end{minted}
	
	\section{Test and debug until correct}
	Run the program and make sure all the tests pass, if not debug until they do. Many of the
	problems you might have had will already have been fixed because of following the "run early, run
	often" approach. But if not, debug until everything works.
	
	\chapter{How To Design Data (HTDD)} \label{ch:htdd}
	Data definitions are a driving element in the design recipes.
	
	A data definition establishes the represent/interpret relationship between information and data:
	\begin{itemize}
		\item Information in the program's domain is represented by data in the program.
		\item Data in the program can be interpreted as information in the program's domain.
	\end{itemize}

	A data definition must describe how to form (or make) data that satisfies the data definition and
	also how to tell whether a data value satisfies the data definition. It must also describe how to
	represent information in the program's domain as data and interpret a data value as information.
	
	So, for example, one data definition might say that numbers are used to represent the Speed of a
	 ball. Another data definition might say that numbers are used to represent the Height of an 
	 airplane. So given a number like 6, we need a data definition to tell us how to interpret it:
	 is it a Speed, or a Height or something else entirely. Without a data definition, the 6 could
	 mean anything.
	\\ \\
	\fbox{
	\parbox{12cm}{
	The first step of the recipe is to identify the inherent structure of the information.

	Once that is done, a data definition consists of four or five elements:
	\begin{enumerate}
		\item A possible \emph{structure definition} (not until compound data)
		\item A \emph{type comment} that defines a new type name and describes how to form data of that type.
		\item An \emph{interpretation} that describes the correspondence between information and data.
		\item One or more \emph{examples} of the data.
		\item A \emph{template} for a 1 argument function operating on data of this type.
	\end{enumerate}
	In the first weeks of the course we also ask you to include a list of the \emph{template rules} used to form the
	template.
	}}
	\pagebreak
	\section{What is the Inherent Structure of the Information?}
	One of the most important points in the course is that:
	\begin{itemize}
		\item the \emph{structure of the information} in the program's domain determines the kind of data definition used,
		\item which in turn determines the \emph{structure of the templates} and helps determine the function examples (check-expects),
		\item and therefore the \emph{structure of much of the final program design}.
	\end{itemize}

	The remainder of this chapter lists in detail different kinds of data definition that are used to
	represent information with different structures. It also shows in detail how to design a
	data definition of each kind. This summary table provides a quick reference to which kind of data
	definition to use for different information structures.
	
	\begin{table}[h]
		\renewcommand{\arraystretch}{1.5}
		\renewcommand{\tabcolsep}{0.2cm}
		\begin{tabularx}{\textwidth}{|X|X|}
			\hline
			\textbf{When the form of the information to be represented...}                     & \textbf{Use a data definition of this kind} \\
			\hline
			is atomic                                                                          &      \nameref{sec:simple_atomic_data}       \\
			\hline
			is numbers within a certain range                                                  &           \nameref{sec:interval}            \\
			\hline
			consists of a fixed number of distinct items                                       &         \nameref{sec:enumerations}          \\
			\hline
			is comprised of 2 or more subclasses, at least one of which is not a distinct item &         \nameref{sec:itemizations}          \\
			\hline
			consists of two or more items that naturally belong together                       &                \nameref{sec:compound_data}                \\
			\hline
			is naturally composed of different parts                                           &      \nameref{sec:ref_other_data_def}       \\
			\hline
			is of arbitrary (unknown) size                                                     &  \nameref{sec:self_or_mut_ref}   \\
			\hline
		\end{tabularx}
		\caption{Types of Data Definition}
	\end{table}
	\pagebreak
	\section{Simple Atomic Data} \label{sec:simple_atomic_data}
	Use simple atomic data when the information to be represented is itself atomic in form, such as
	the elapsed time since the start of the animation, the x coordinate of a car or the name of a cat.
	
	\begin{minted}{racket}
	;; Time is Natural
	;; interp. number of clock ticks 
	;; since start of game
	(define START-TIME 0)
	(define OLD-TIME 1000)
	
	#;
	(define (fn-for-time t)
		(... t))
	
	;; Template rules used:
	;; - atomic non-distinct: Natural
	\end{minted}
	
	\subsection*{Forming the Template}
	As noted below the template, it is formed according to the Data Driven Templates recipe using the
	right hand column of the atomic non-distinct rule.
	
	\subsection*{Guidance on Data Examples and Function Example/Tests}
	One or two data examples are usually sufficient for simple atomic data.
	
	When creating example/tests for a specific function operating on simple atomic data at least one
	test case will be required. Additional tests are required if there are multiple cases involved. If the
	function produces Boolean there needs to be at least a true and false test case. Also be on the
	lookout for cases where a number of some form is an interval in disguise, for example given a type
	comment like Countdown is Natural, in some functions 0 is likely to be a special case.
	\pagebreak
	\section{Intervals} \label{sec:interval}
	Use an interval when the information to be represented is numbers within a certain range.
	Integer[0, 10] is all the integers from 0 to 10 inclusive; Number[0, 10) is all the numbers
	from 0 inclusive to 10 exclusive. The notation is that [ and ] mean that the end of the interval
	includes the end point; ( and ) mean that the end of the interval does not include the end point.
	
	Intervals often appear in itemizations, but can also appear alone, as in:
	
	\begin{minted}{racket}
	;; Countdown is Integer[0, 10]
	;; interp. the number of seconds
	;; remaining to liftoff
	(define C1 10) ; start
	(define C2 5)  ; middle
	(define C3 0)  ; end
	
	#;
	(define (fn-for-countdown cd)
		(... cd))
		
	;; Template rules used:
	;; - atomic non-distinct: Integer[0, 10]
	\end{minted}
	
	\subsection*{Forming the Template}
	As noted below the template, it is formed according to the Data Driven Templates recipe using the
	right hand column of the atomic non-distinct rule.
	
	\subsection*{Guidance on Data Examples and Function Example/Tests}
	For data examples provide sufficient examples to illustrate how the type represents information.
	The three data examples above are probably more than is needed in that case.
	
	When writing tests for functions operating on intervals be sure to test closed boundaries as well
	as midpoints. As always, be sure to include enough tests to check all other points of variance in
	behaviour across the interval.
	\pagebreak
	\section{Enumerations} \label{sec:enumerations}
	Use an enumeration \emph{when the information to be represented consists of a fixed number of
	distinct items}, such as colors, letter grades etc. The data used for an enumeration could in
	principle be anything - strings, integers, images even. But we always use strings. In the case of
	enumerations it is sometimes redundant to provide an interpretation and nearly always redundant
	to provide examples. The example below includes the interpretation but not the examples.
	
	\begin{minted}{racket}
	;; LightState is one of:
	;; - "red"
	;; - "yellow"
	;; - "green"
	;; interp. the color of a traffic light
	
	;; <examples are redundant for enumerations>
	
	#;
	(define (fn-for-light-state ls)
		(cond [(string=? "red" ls) (...)]
		      [(string=? "yellow" ls) (...)]
		      [(string=? "green" ls) (...)]))
			
	;; Template rules used:
	;; - one of: 3 cases
	;; - atomic distinct: "red"
	;; - atomic distinct: "yellow"
	;; - atomic distinct: "green"
	\end{minted}
	
	\subsection*{Forming the Template}
	As noted below the template, it is formed according to the Data Driven Templates recipe as	follows:
	
	First, LightState is an enumeration with 3 cases, so the one of rule says to use a cond with 3	cases:
	
	\begin{minted}{racket}
	(define (fn-for-tlcolor ls)
		(cond [Q1 A1]
		      [Q2 A2]
		      [Q3 A3]))
	\end{minted}
	
	In the first clause, "red" is a distinct atomic value, so the cond question column of the atomic
	distinct rule says Q1 should be (string=? ls "red"). The cond answer column says A1 should
	be (...). So we have:
	
	\begin{minted}{racket}
	(define (fn-for-light-state ls)
		(cond [(string=? "red" ls) (...)]
		      [Q2 A2]
		      [Q3 A3]))
	\end{minted}
	
	Then "yellow" and "green" are also distinct atomic values, so the final template is:
	
	\begin{minted}{racket}
	(define (fn-for-light-state ls)
		(cond [(string=? "red" ls)(...)]
		      [(string=? "yellow" ls) (...)]
		      [(string=? "green" ls) (...)]))
	\end{minted}
	
	\subsection*{Guidance on Data Examples and Function Example/Tests}
	Data examples are redundant for enumerations.
	
	Functions operating on enumerations should have (at least) as many tests as there are cases in
	the enumeration.
	
	\subsection*{Large Enumerations}
	Some enumerations contain a large number of elements. A canonical example is KeyEvent, which
	is provided as part of big-bang. KeyEvent includes all the letters of the alphabet as well as other
	keys you can press on the keyboard. It is not necessary to write out all the cases for such a data
	definition. Instead write one or two, as well as a comment saying what the others are, where they
	are defined etc.
	
	Defer writing templates for such large enumerations until a template is needed for a specific
	function. At that point include the specific cases that function cares about. Be sure to include an
	else clause in the template to handle the other cases. As an example, some functions operating
	on KeyEvent may only care about the space key and just ignore all other keys, the following
	would be an appropriate template for such functions.
	
	\begin{minted}{racket}
	#;
	(define (fn-for-key-event kevt)
		(cond [(key=? " " kevt) (...)]
		      [else
			  (...)]))
			  
	;; Template formed using the large
	;; enumeration special case
	\end{minted}
	
	The same is true of writing tests for functions operating on large enumerations. All the specially
	handled cases must be tested, in addition one more test is required to check the else clause.
	\pagebreak
	\section{Itemizations} \label{sec:itemizations}
	An itemization describes \emph{data comprised of 2 or more subclasses, at least one of which is not a
	distinct item}. (C.f.\ enumerations, where the subclasses are \emph{all} distinct items.) In an itemization
	the template is similar to that for enumerations: a cond with one clause per subclass. In cases
	where the subclass of data has its own data definition the answer part of the cond clause
	includes a call to a helper template, in other cases it just includes the parameter.
	
	\begin{minted}{racket}
	;; Bird is one of:
	;; - false
	;; - Number
	;; interp. false means no bird,
	;; number is x position of bird
	
	(define B1 false)
	(define B2 3)
	
	#;
	(define (fn-for-bird b)
		(cond [(false? b) (...)]
		      [(number? b) (... b)]))
		      
	;; Template rules used:
	;; - one of: 2 cases
	;; - atomic distinct: false
	;; - atomic non-distinct: Number
	\end{minted}
	
	\subsection*{Forming the Template}
	As noted below the template, it is formed according to the Data Driven Templates recipe using the
	\emph{one-of rule}, the \emph{atomic distinct rule} and the \emph{atomic non-distinct rule} in order.
	
	\subsection*{Guidance on Data Examples and Function Example/Tests}
	As always, itemizations should have enough data examples to clearly illustrate how the type
	represents information.
	
	Functions operating on itemizations should have at least as many tests as there are cases in the
	itemizations. If there are intervals in the itemization, then there should be tests at all points of
	variance in the interval. In the case of adjoining intervals it is critical to test the boundaries.
	
	\subsection*{Itemization of Intervals}
	A common case is for the itemization to be comprised of 2 or more intervals. In this case functions
	operating on the data definition will usually need to be tested at all the boundaries of closed
	intervals and points between the boundaries.
	
	\begin{minted}{racket}
	;;; Reading is one of:
	;; - Number[> 30]
	;; - Number(5, 30]
	;; - Number[0, 5]
	;; interp. distance in cm from bumper to obstacle
	;; Number[> 30] 	is considered "safe"
	;; Number(5, 30]	is considered "warning"
	;; Number[0, 5] 	is considered "dangerous"
	
	(define R1 40)
	(define R2 .9)
	
	#;
	(define (fn-for-reading r)
		(cond [(< 30 r) (... r)]
		      [(and (< 5 r) (<= r 30)) (... r)]
		      [(<= 0 r 5) (... r)]))
		      
	;; Template rules used:
	;; one-of: 3 cases
	;; atomic non-distinct: Number[>30]
	;; atomic non-distinct: Number(5, 30]
	;; atomic non-distinct: Number[0, 5]
	\end{minted}
	
	As noted below the template, it is formed according to the Data Driven Templates recipe using the
	\emph{one-of rule}, followed by 3 uses of the \emph{atomic non-distinct rule}.
	\pagebreak
	\section{Compound data (structures)} \label{sec:compound_data}
	Use structures when two or more values naturally belong together. The define-struct goes at the
	beginning of the data definition, before the types comment.
	
	\begin{minted}{racket}
	(define-struct ball (x y))
	;; Ball is (make-ball Number Number)
	;; interp. a ball at position x, y
	
	(define BALL-1 (make-ball 6 10))
	
	#;
	(define (fn-for-ball b)
		(... (ball-x b)      ;Number
		     (ball-y b)))    ;Number
		
	;; Template rules used:
	;; - compound: 2 fields
	\end{minted}
	
	The template above is formed according to the Data Driven Templates recipe using the compound
	rule. Then for each of the selectors, the result type of the selector (Number in the case of ball-x
	and ball-y) is used to decide whether the selector call itself should be wrapped in another
	expression. In this case, where the result types are primitive, no additional wrapping occurs.
	C.f.\ cases below when the reference rule applies.
	
	\subsection*{Guidance on Data Examples and Function Example/Tests}
	For compound data definitions it is often useful to have numerous examples, for example to
	illustrate special cases. For a snake in a snake game you might have an example where the snake
	is very short, very long, hitting the edge of a box, touching food etc. These data examples can
	also be useful for writing function tests because they save space in each check-expect.
	\pagebreak
	\section{References to other data definitions} \label{sec:ref_other_data_def}
	Some data definitions contain references to other data definitions you have defined (non-primitive
	data definitions). One common case is for a compound data definition to comprise other named
	data definitions. (Or, once lists are introduced, for a list to contain elements that are described by
	another data definition. In these cases the template of the first data definition should contain calls
	to the second data definition's template function wherever the second data appears. For example:
	
	\begin{minted}{racket}
	;---assume Ball is as defined above---
	
	(define-struct game (ball score))
	;; Game is (make-game Ball Number)
	;; interp. the current ball and score of the game
	
	(define GAME-1 (make-game (make-ball 1 5) 2))
	
	#;
	(define (fn-for-game g)
		(... (fn-for-ball (game-ball g))
		     (game-score g)))     ;Number
		     
	;; Template rules used:
	;; - compound: 2 fields
	;; - reference: ball field is Ball
	\end{minted}
	
	In this case the template is formed according to the Data Driven Templates recipe by first using
	the compound rule. Then, since the result type of (game-ball g) is Ball, the reference rule is
	used to wrap the selector so that it becomes (fn-for-ball (game-ball g)). The call to
	game-score is not wrapped because it produces a primitive type.
	
	\subsection*{Guidance on Data Examples and Function Example/Tests}
	For data definitions involving references to non-primitive types the data examples can sometimes
	become quite long. In these cases it can be helpful to define well-named constants for data examples 
	for the referred to type and then use those constants in the referring from type. For example:
	
	\begin{minted}{racket}
	;...in the data definition for Drop...
	(define DTOP (make-drop 10 0))              ;top of screen
	(define DMID (make-drop 20 (/ HEIGHT 2)))   ;middle of screen
	(define DBOT (make-drop 30 HEIGHT))         ;at bottom edge
	(define DOUT (make-drop 40 (+ HEIGHT 1)))   ;past bottom edge	
	
	;...in the data definition for ListOfDrop...
	(define LOD1 empty)
	(define LOD-ALL-ON (cons DTOP (cons DMID empty)))
	(define LOD-ONE-ABOUT-TO-LEAVE (cons DTOP 
	                                (cons DMID (cons DBOT empty))))
	(define LOD-ONE-OUT-ALREADY (cons DTOP (cons DMID 
	                                (cons DBOT (cons DOUT
                                 	 empty)))))
	\end{minted}
	
	In the case of references to non-primitive types the function operating on the referring type (i.e.\ 
	ListOfDrop) will end up with a call to a helper that operates on the referred to type (i.e.\ Drop).
	Tests on the helper function should fully test that function, tests on the calling function may
	assume the helper function works properly.
	\pagebreak
	\section{Self-referential or mutually referential} \label{sec:self_or_mut_ref}
	When the \emph{information in the program's domain is of arbitrary size}, a well-formed self-referential
	(or mutually referential) data definition is needed.
	\\ \\
	In order to be well-formed, a self-referential data definition must:
	
	\begin{itemize}
		\item have at least one case without self reference (the base case(s))
		\item have at least one case with self reference
	\end{itemize}

	The template contains a base case corresponding to the non-self-referential clause(s) as well as
	one or more natural recursions corresponding to the self-referential clauses.
	
	\begin{minted}{racket}
	;; ListOfString is one of:
	;; - empty
	;; - (cons String ListOfString)
	;; interp. a list of strings
	
	(define LOS-1 empty)
	(define LOS-2 (cons "a" empty))
	(define LOS-3 (cons "b" (cons "c" empty)))
	
	#;
	(define (fn-for-los los)
		(cond [(empty? los) (...)]              ;BASE CASE
		      [else (... (first los)            ;String
		            (fn-for-los (rest los)))])) ;NATURAL RECURSION
	;;	         /
	;;	        /
	;;	  COMBINATION
	;; Template rules used:
	;; - one of: 2 cases
	;; - atomic distinct: empty
	;; - compound: (cons String ListOfString)
	;; - self-reference: (rest los) is ListOfString
	\end{minted}
	
	In some cases a types comment can have both self-reference and reference to another type.
	
	\begin{minted}{racket}
	(define-struct dot (x y))
	;; Dot is (make-dot Integer Integer)
	;; interp. A dot on the screen, w/ x and y coordinates.
	
	(define D1 (make-dot 10 30))
	
	#;
	(define (fn-for-dot d)
	        (... (dot-x d)          ;Integer
	             (dot-y d)))        ;Integer
	
	;; Template rules used:
	;; - compound: 2 fields
	
	
	;; ListOfDot is one of:
	;; - empty
	;; - (cons Dot ListOfDot)
	;; interp. a list of Dot
	
	(define LOD1 empty)
	(define LOD2 (cons (make-dot 10 20) 
	                    (cons (make-dot 3 6) empty)))
	
	#;
	(define (fn-for-lod lod)
		(cond [(empty? lod) (...)]
		      [else
		          (... (fn-for-dot (first lod))
		               (fn-for-lod (rest lod)))]))
	
	;; Template rules used:
	;; - one of: 2 cases
	;; - atomic distinct: empty
	;; - compound: (cons Dot ListOfDot)
	;; - reference: (first lod) is Dot
	;; - self-reference: (rest lod) is ListOfDot
	\end{minted}
	
	\subsection*{Guidance on Data Examples and Function Example/Tests}
	When writing data and function examples for self-referential data definitions always put the base
	case first. Its usually trivial for data examples, but many function tests don't work properly if the
	base case isn't working properly, so testing that first can help avoid being confused by a failure in
	a non base case test. Also be sure to have a test for a list (or other structure) that is at least 2
	long.
	
	\chapter{How To Design Worlds (HtDW)} \label{ch:htdw}
	The How to Design Worlds process provides guidance for designing interactive world programs
	using big-bang. While some elements of the process are tailored to big-bang, the process can
	also be adapted to the design of other interactive programs. The wish-list technique can be used
	in any multi-function program.
	\\ \\
	\fbox{
		\parbox{12cm}{
			World program design is divided into two phases, each of which has sub-parts:
			
			\begin{enumerate}
				\item Domain analysis (use a piece of paper!)
				\begin{enumerate}
					\item Sketch program scenarios
					\item Identify constant information \label{itm:constant}
					\item Identify changing information \label{itm:changing}
					\item Identify big-bang options \label{itm:big_bang}
				\end{enumerate}
				\item Build the actual program
				\begin{enumerate}
					\item Constants (based on \autoref*{itm:constant} above)
					\item Data definitions using \nameref{ch:htdd} (based on \autoref*{itm:changing} above) \label{itm:data_def}
					\item Functions using \nameref{ch:htdf}
					\begin{enumerate}
						\item main first (based on \autoref*{itm:changing}, \autoref*{itm:big_bang} and \autoref*{itm:data_def} above)
						\item wish list entries for big-bang handlers
					\end{enumerate}
					\item Work through wish list until done
				\end{enumerate}
			\end{enumerate}
			}
		}
	\pagebreak
	\section{Phase 1: Domain Analysis}
	Do a domain analysis by hand-drawing three or more pictures of what the world program will look
	like at different stages when it is running.
	
	Use this picture to identify constant information such as the height and width of screen, color of 
	the background, the background image itself, the length of a firework's fuse, the image for a
	moving cat and so on.
	
	Also identify changing information such as the position of a firework, the color of a light, the
	number in countdown etc.
	
	Identify which big-bang options the program needs.
	
	\begin{table}[h]
		\renewcommand{\arraystretch}{2.5}
		\renewcommand{\tabcolsep}{0.2cm}
		\begin{tabularx}{\textwidth}{|X|X|}
			\hline
			\textbf{If your program needs to:}     & \textbf{Then it needs this option:} \\
			\hline
			change as time goes by (nearly all do) & on-tick                             \\
			\hline
			display something (nearly all do)      & to-draw                             \\
			\hline
			change in response to key presses      & on-key                              \\
			\hline
			change in response to mouse activity   & on-mouse                            \\
			\hline
			stop automatically                     & stop-when                           \\
			\hline
		\end{tabularx}
		\caption{Big-Bang Options}
	\end{table}

	(There are several more options to big-bang. Look in the DrRacket help desk under big-bang for
	a complete list.)
	\pagebreak
	\section{Phase 2: Building the actual program}
	Structure the actual program in four parts:
	\begin{itemize}
		\item Requires followed by one line summary of program's behavior
		\item Constants
		\item Data definitions
		\item Functions
	\end{itemize}

	The program should begin with whatever require declarations are required. For a program using
	big-bang this is usually a require for 2htdp/universe to get big-bang itself and a require for
	2htdp/image to get useful image primitives. This is followed by a short summary of the
	program's behavior (ideally 1 line).
	
	The next section of the file should define constants. These will typically come directly from the
	domain analysis.
	
	This is followed by data definitions. The data definitions describe how the world state - the
	changing information identified during the analysis - will be represented as data in the program.
	Simple world programs may have just a single data definition. More complex world programs have
	a number of data definitions.
	
	The functions section should begin with the main function which uses big-bang with the
	appropriate options identified during the analysis. After that put the more important functions first
	followed by the less important helpers. Keep groups of closely related functions together.
	
	\section{Template for a World Program}
	A useful template for a world program, including a template for the main function and wish list
	entries for tick-handler and to-draw handler is as follows. To use this template replace WS with
	the appropriate type for your changing world state. You may want to give the handler functions
	more descriptive names and you should definitely give them all a more descriptive purpose.
	
	\begin{minted}{racket}
	(require 2htdp/image)
	(require 2htdp/universe)
	
	;; My world program (make this more specific)
	
	;; =================
	;; Constants:
	
	;; =================
	;; Data definitions:
	;; WS is ... (give WS a better name)
	
	;; =================
	;; Functions:
	;; WS -> WS
	;; start the world with ...
	;;
	(define (main ws)
		(big-bang ws        ;WS
		  (on-tick tock)    ;WS -> WS
		  (to-draw render)  ;WS -> Image
		  (stop-when ...)   ;WS -> Boolean
		  (on-mouse ...)    ;WS Integer Integer MouseEvent -> WS
		  (on-key ...)))    ;WS KeyEvent -> WS
	
	;; WS -> WS
	;; produce the next ...
	;; !!!
	(define (tock ws) ...)
	
	;; WS -> Image
	;; render ...
	;; !!!
	(define (render ws) ...)
	\end{minted}
	
	Depending on which other big-bang options you are using you would also end up with wish list
	entries for those handlers. So, at an early stage a world program might look like this:
	
	\begin{minted}{racket}
	(require 2htdp/universe)
	(require 2htdp/image)
	
	;; A cat that walks across the screen.
	
	;; =================
	;; Constants:
	
	(define WIDTH 200)
	(define HEIGHT 200)
	(define CAT-IMG (circle 10 "solid" "red")) ; a cat
	(define MTS (empty-scene WIDTH HEIGHT))
	
	;; =================
	;; Data definitions:
	
	;; Cat is Number
	;; interp. x coordinate of
	;; cat (in screen coordinates)
	(define C1 1)
	(define C2 30)
	
	#;
	(define (fn-for-cat c)
		(... c))
	
	;; =================
	;; Functions:
	
	;; Cat -> Cat
	;; start the world with initial state c,
	;; for example: (main 0)
	(define (main c)
		(big-bang c           ;Cat
		  (on-tick tock)      ;Cat -> Cat
		  (to-draw render)))  ;Cat -> Image
		
	;; Cat -> Cat
	;; Produce cat at next position
	;!!!
	(define (tock c) 1) ;stub
	
	;; Cat -> Image
	;; produce image with CAT-IMG placed on
	;; MTS at proper x, y position
	; !!!
	(define (render c) MTS)
	\end{minted}
	
	Note that we are maintaining a wish list of functions that need to be designed. The way to
	maintain the wish list is to just write a signature, purpose and stub for each wished-for function,
	also label the wish list entry with !!! or some other marker that is easy to search for. That will
	help you find your unfilled wishes later.
	
	Forming wish list entries this way is enough for main (or other functions that call a wished for
	function) to be defined without error. But of course main (and other such functions) will not run
	properly until the wished for functions are actually completely designed.
	
	As you design the program remember to run early and run often. The sooner you can run the
	program after writing anything the sooner you can find any small mistakes that might be in it.
	Fixing the small mistakes earlier makes it easier to find any harder mistakes later.
	
	\section{Key and Mouse Handlers}
	The on-key and on-mouse handler function templates are handled specially. The on-key
	function is templated according to its second argument, a KeyEvent, using the large enumeration
	rule. The on-mouse function is templated according to its MouseEvent argument, also using the
	large enumeration rule. So, for example, for a key handler function that has a special behaviour
	when the space key is pressed but does nothing for any other key event the following would be
	the template:
	
	\begin{minted}{racket}
	(define (handle-key ws ke)
		(cond [(key=? ke " ") (... ws)]
		      [else
		        (... ws)]))
	\end{minted}
	
	Similarly the template for a mouse handler function that has special behavior for mouse clicks but
	ignores all other mouse events would be:
	
	\begin{minted}{racket}
	(define (handle-mouse ws x y me)
		(cond [(mouse=? me "button-down") (... ws x y)]
		      [else
		        (... ws x y)]))
	\end{minted}
	
	For more information on the KeyEvent and MouseEvent large enumerations see the DrRacket
	help desk.
	
	\chapter{Data Driven Templates} \label{ch:data_driv_temp}
	Templates are the core structure that we know a function must have, independent of the details
	of its definition. In many cases the template for a function is determined by the type of data the
	function consumes. We refer to these as data driven templates. The recipe below can be used to
	produce a data driven template for any type comment.
	\\ \\
	For a given type TypeName the data driven template is:
	
	\begin{minted}{racket}
	(define (fn-for-type-name x)
		<body>)
	\end{minted}
	
	Where x is an appropriately chosen parameter name (often the initials of the type name) and the
	body is determined according to the table below. To use the table, start with the type of the
	parameter, i.e. TypeName, and select the row of the table that matches that type. The first row
	matches only primitive types, the later rows match parts of type comments.
	
	(Note that when designing functions that consume additional atomic parameters, the name of
	that parameter gets added after every $\ldots$ in the template. Templates for functions with
	additional complex parameters are covered in Functions on 2 One-Of Data.)
	\pagebreak
	\renewcommand{\arraystretch}{1.5}
	\begin{longtable}{|p{1.5 in}|p{1.5 in}|p{1.5 in}|}
		\hline
		\bfseries Type of data & \bfseries Cond question (if applicable) & \bfseries Body or cond answer (if applicable) 
		\endhead \hline
		
		\hline
		    \emph{Atomic Non-Distinct}
		    \begin{itemize}
		     	\item Number
		     	\item String
		     	\item Boolean
		     	\item Image
		     	\item interval like Number[0, 10)
		     	\item etc.
		    \end{itemize} 
	     	& Appropriate predicate
	     	\begin{itemize}
			     \item \mintinline{racket}{(number? x)}
			     \item \mintinline{racket}{(string? x)}
			     \item \mintinline{racket}{(boolean? x)}
			     \item \mintinline{racket}{(image? x)}
			     \item 
			     \begin{minted}[linenos=false]{racket}
			     (and (<= 0 x)
			          (< x 10))
			     \end{minted}
			     \item etc.
	     	\end{itemize} 
     		& Expression that operates on the parameter. \mint[linenos=false]{racket}{(... x)} \\
		\hline
			\emph{Atomic Distinct Value}
			\begin{itemize}
				\item "red"
				\item false
				\item empty
				\item etc.
			\end{itemize}
			& Appropriate predicate
			\begin{itemize}
				\item \mint[linenos=false]{racket}{(string=? x "red")}
				\item \mintinline{racket}{(false? x)}
				\item \mintinline{racket}{(empty? x)}
				\item etc.
			\end{itemize}
			& Since value is distinct, parameter does not appear. \mint[linenos=false]{racket}{(...)} \\
		\hline
			\emph{One Of}
			\begin{itemize}
				\item enumerations
				\item itemizations
			\end{itemize} 
			& 
			& Cond with one clause per subclass of one of.
			\begin{minted}[linenos=false]{racket}
            ; Q - question
            ; A - answer
            (cond [<Q1> <A1>]
                  [<Q2> <A2>])
            \end{minted}
			Where each question and answer expression is
			formed by following the rule in the question or
			answer column of this table for the corresponding
			case. A detailed derivation of a template for a one-of
			type appears below.
			\newline \newline
			It is permissible to use else for the last question for
			itemizations and large enumerations. Normal
			enumerations should not use else.
			\newline \newline
			Note that in a mixed data itemization, such as:
			\begin{minted}[linenos=false]{racket}
			;; Measurement is one of:
			;;- Number[-10, 0)
			;;- true
			;;- Number(0, 10]
			\end{minted}
			the cond questions must be \emph{guarded} \\ 
		\hline 
			&
			&  with an appropriate type predicate. In particular, the first
			cond question for Measurement must be
			\begin{minted}[linenos=false]{racket}
		    (and (number? m)
		         (<= -10 m)
		         (< m 0))
			\end{minted}
			where the call to \mintinline{racket}{number?} guards the calls to \mintinline{racket}{<=} and
			\mintinline{racket}{<}. This will protect \mintinline{racket}{<=} and \mintinline{racket}{<} from ever receiving true
			as an argument.\\
		\hline
			\emph{Compound}
			\begin{itemize}
				\item Position
				\item Firework
				\item Ball
				\item cons
				\item etc.
			\end{itemize}
			& Predicate from structure
			\begin{itemize}
				\item \mint[linenos=false]{racket}{(posn? x)}
				\item \mint[linenos=false]{racket}{(firework? x)}
				\item \mint[linenos=false]{racket}{(ball? x)}
				\item \mintinline{racket}{(cons? x)} (often just \mintinline{racket}{else})
				\item etc.
			\end{itemize}
			& All selectors.
			\begin{itemize}
				\item \mint[linenos=false]{racket}{(... (posn-x x) (posn-y x))}
				\item \mint[linenos=false]{racket}{(... (firework-y x) (firework-color x))}
				\item \mint[linenos=false]{racket}{(... (ball-x x) (ball-dx x))}
				\item \mint[linenos=false]{racket}{(... (first x) (rest x))}
				\item etc.
			\end{itemize}
			Then consider the \\
		\hline  
			&
			&result type of each selector call
			and wrap the accessor expression appropriately
			using the table with that type. So for example, if after
			adding all the selectors you have: 
			\begin{minted}[linenos=false]{racket}
			(... (game-ball g) ;produces Ball
			  (game-paddle g)) ;produces Paddle
			\end{minted}
			Then, because both Ball and Paddle are non-primitive
			types (types that you yourself defined in a data
			definition) the reference rule (immediately below)
			says that you should add calls to those types'
			template functions as follows:
			\begin{minted}[linenos=false]{racket}
			(... (fn-for-ball (game-ball g))
			  (fn-for-paddle (game-paddle g)))
			\end{minted}
			\\
		\hline
			\emph{Other Non-Primitive Type Reference} 
			& Predicate, usually from structure definition
			\begin{itemize}
				\item \mint[linenos=false]{racket}{(firework? x)}
				\item \mint[linenos=false]{racket}{(person? x)}
			\end{itemize} 
			& Call to other type's template function
			\begin{itemize}
				\item \mint[linenos=false]{racket}{(fn-for-firework x)}
				\item \mint[linenos=false]{racket}{(fn-for-person x)}
			\end{itemize} \\
		\hline
			\emph{Self Reference} 
			&
			& Form natural recursion with call to this type's
			template function: 
			\mint[linenos=false]{racket}{(fn-for-los (rest los))}\\
		\hline
			\emph{Mutual Reference} \newline \newline
			Note: form and group all templates
			in mutual reference cycle
			together.
			&
			& Call to other type's template function:
			\begin{minted}[linenos=false]{racket}
			(fn-for-lod (dir-subdirs d)
			(fn-for-dir (first lod))
			\end{minted}
			\\
		\hline
		\caption{Data Driven Templates}
	\end{longtable}
	\pagebreak
	\section{Producing the Template for an Example - One Of Type}
	In many cases more than one of the above rules will apply to a single template. Consider this type
	comment:
	
	\begin{minted}{racket}
	;; Clock is one of:
	;; - Natural
	;; - false
	\end{minted}
	and the step-by-step construction of the template for a function operating on Clock.
	\begin{enumerate}
		\item Clock is a one of type with two subclasses (one of which is not distinct
		making it an itemization). The one of rule tells us to use a cond. The cond
		needs one clause for each subclass of the itemization.
		\begin{minted}{racket}
		(define (fn-for-clock c)
			(cond [Q A]
			      [Q A]))
			      
		;; Template rules used:
		;; - one of: 2 cases
		\end{minted}

		\item The cond questions need to identify each subclass of data. The cond answers
		need to follow templating rules for that subclasses data. In the first subclass,
		Natural is a non-distinct type; the atomic non-distinct rule tells us the
		question and answer as shown to the left.
		\begin{minted}{racket}
		(define (fn-for-clock c)
			(cond [(number? c) (... c)]
			      [Q A]))
			      
		;; Template rules used:
		;; - one of: 2 cases
		;; - atomic non-distinct: Natural
		\end{minted}
		
		\item In the second case false is an atomic distinct type, so the atomic-distinct
		rule gives us the question and answer. Since the second case is also the last
		case we can use else for the question.
		\begin{minted}{racket}
		(define (fn-for-clock c)
			(cond [(number? c) (... c)]
			      [else
			        (...)]))
		;; Template rules used:
		;; - one of: 2 cases
		;; - atomic non-distinct: Natural
		;; - atomic distinct: false
		\end{minted}
	\end{enumerate}

	\section{Templates for Mutually Referential Types}
	The previous example doesn't cover the mutual-reference rule, which says that in the case of
	mutually-referential data definitions, when you template one function in the mutual-reference
	cycle you should immediately template all the functions in the mutual-reference cycle. So, for
	example, given:
	\begin{minted}{racket}
	(define-struct person (name subs))
	;; Person is (make-person String ListOfPerson)
	
	;; ListOfPerson is one of:
	;; - empty
	;; - (cons Person ListOfPerson)
	\end{minted}
	Then if you need a template for a function operating on a Person (or a function operating on a
	ListOfPerson) you should immediately write a template for both functions, resulting in
	something like this:
	\begin{minted}{racket}
	#;
	(define (fn-for-person p)
		(... (person-name p)
		     (fn-for-lop (person-subs p)))) ;MRMR
	
	#;
	(define (fn-for-lop lop)
		(cond [(empty? lop) ...]
		      [else
		        (... (fn-for-person (first lop)) ;MRMR
	        	     (fn-for-lop (rest lop)))])) ;NRSF
	        	     
	; - MRMR: mutual recursion from mutual-reference
	; - NRSR: natural recursion from self-reference
	\end{minted}
	
	(Note that producing that template will also involve using the atomic-distinct, atomic, one-of and
	compound rules.)
	
	As with self-reference, its a good idea to draw a mutual-reference line on the data definition and
	ensure you have corresponding mutual recursion lines in your templates.
	
	\section{Testing}
	The principles above can also be used to understand how many tests a data definition implies.
	Simply put, the set of tests/examples should cover all cases, call all helper functions, involve all
	selectors, and avoid duplicated values.
	
	\section{Additional Design Rules for Helpers}
	During coding three additional guidelines suggest situations under which a helper function should
	be added:
	\begin{enumerate}
		\item Use a separate function for each difference between quantities in a problem.
		\item If a subtask requires operating on arbitrary sized data a helper function must be used.
		\item If a subtask involves special domain knowledge a helper function should be used.
		\item In addition always keep the "one task per function" goal in mind. If part of a function you are
		designing seems to be a well-defined subtask put that into a helper function.
	\end{enumerate}

	\chapter{Functions On 2 One-Of Data} \label{ch:fun_2_1_of_data}
	This chapter outlines the variations in the normal HtDF recipe when designing a function that
	consumes 2 data that have a one-of in their type comments. Examples of functions for which this
	applies include functions with the following signatures:
	\begin{minted}{racket}
	;; ListOfString ListOfString -> Boolean
	;; ListOfString ListOfString -> ListOfString
	;; ListOfString BinaryTree -> Boolean
	;; ListOfNatural FamilyTree -> ListOfString
	\end{minted}
	
	For the purpose of this explanation, assume that the goal is to design a function that consumes
	two ListOfString and produces true if the strings in the first list are equal to the corresponding
	strings in the second list. If that is true and the second list is longer than the first the function
	should produce true; if the second list is shorter it always produces false.
	
	The first three steps of the recipe - signature, purpose and stub - are unchanged.
	
	\begin{minted}{racket}
	;; ListOfString ListOfString -> Boolean
	;; produce true if lsta is a prefix of lstb
	
	(define (prefix=? lsta lstb) false)
	\end{minted}
	
	But at this point the next step is to form a \emph{cross-product of type comments} table as follows. The
	row labels of the table are the cases of the one-of type comment for one argument (perhaps the
	first), and the column labels are the cases of the one-of type comment for the other argument.
	\clearpage
	\begin{minted}{racket}
	;; CROSS PRODUCT OF TYPE COMMENTS TABLE
	;;                               lstb
	;;                     empty          (cons String LOS)
	;;                                 |
	;; l  empty                        |
	;; s                   --------------------------------
	;; t  (cons String LOS)            |
	;; a                               |
	\end{minted}
	
	In this case, where both arguments have 2 cases in their one-of type comments, the cross-
	product formed has 4 cells (2 * 2 = 4). The next step of the process is to use the cross product
	table to help form at least as many tests as there are cells. The upper left cell describes a
	scenario where both lsta and lstb are empty. The lower left cell is where lsta is non-empty but
	lstb is empty and so on. The lower right cell is where both lists are non-empty, and this case
	requires more than one test. So we end up with:
	
	\begin{minted}{racket}
	(check-expect (prefix=? empty empty) true)
	(check-expect (prefix=? empty (list "a" "b")) true)
	(check-expect (prefix=? (list "a") empty) false)
	
	(check-expect (prefix=? (list "a") (list "b")) false)
	(check-expect (prefix=? (list "a") (list "a")) true)
	(check-expect (prefix=? (list "a" "b") (list "a" "x"))
	               false)
	(check-expect (prefix=? (list "a" "b") (list "a" "b"))
	               true)
	\end{minted}
	
	We can now use the tests to help fill in the contents of the table cells, indicating what the function
	should do in each case. Any cell requiring a more complex code answer, such as the lower right
	one in this case, need not be coded perfectly with correct syntax, but should give a good idea of
	what the code must do.
	
	\begin{minted}{racket}
	;; CROSS PRODUCT OF TYPE COMMENTS TABLE
	;;                            lstb
	;;                       empty    (cons String LOS)
	;;                              |
	;; l  empty              true   | true   
	;; s                     --------------------------------
	;; t  (cons String LOS)  false  | (and
	;; a                            |   <firsts are string=?>
	;;                              |   <rests are prefix=?>)
	\end{minted}
	
	Now comes the most fun step. We look for a way to simplify the table by identifying cells that
	have the same answer. In this case the entire first row produces true, so we can simplify the
	table by combining the two cells in the first row into a single cell:
	
	\begin{minted}{racket}
	;; CROSS PRODUCT OF TYPE COMMENTS TABLE
	;;                            lstb
	;;                       empty    (cons String LOS)
	;;                              |
	;; l  empty                    true      
	;; s                     --------------------------------
	;; t  (cons String LOS)  false  | (and
	;; a                            |   <firsts are string=?>
	;;                              |   <rests are prefix=?>)
	\end{minted}
	
	Now we are almost done. The next step is to code the function body directly from the table. This
	in effect intertwines templating with coding of details. Because the simplified table has only three
	cells, we know that the body of the function will be a three case cond. For the first question we
	always pick the largest cell, in this case the top row. The question needs to be true of the entire
	combined cell, so in this case the question is (empty? lsta). The answer in this case is just
	true.
	
	For the next case of the cond we pick the lower left cell. At this point, we know that the top row
	has been handled, so we only need a question that distinguishes the remaining cells apart. In this
	case (empty? lstb) distinguishes the lower left from the lower right cell. The answer in this
	case is false.
	
	In larger tables this process continues until you get to the last cell, at which point the question
	can be else.
	
	In cells that involve natural recursion, the natural recursion can be formed by applying the normal
	rules for handling self-reference. In this case the template for the cond answer in the third cond
	case is:
	
	\begin{minted}{racket}
	(... (first lsta)
	     (first lstb)
	     (prefix=? (rest lsta) ...)
	     (prefix=? ... (rest lstb)))
	\end{minted}
	
	After filling that in we end up with:
	
	\begin{minted}{racket}
	(define (prefix=? lsta lstb)
		(cond [(empty? lsta) true]
		      [(empty? lstb) false]
		      [else
		        (and (string=? (first lsta) (first lstb))
		        (prefix=? (rest lsta)
		                  (rest lstb)))]))
	\end{minted}
	
	In general, when designing a function on 2 one of data it is a good idea to include the cross-product table in the design.
	
	\chapter{Control Driven Templates}
	\section{Function Composition} \label{sec:fun_comp}
	Use function composition when a function must perform two or more distinct and complete
	operations on the consumed data. For example:
	\begin{itemize}
		\item A function that must sort and layout a list of images. First it must sort the complete list and then
		lay it out. It cannot sort and layout each image one at a time.
		\item A function that must advance a list of raindrops and then remove the ones that have left the
		screen. First it must advance all the drops and then remove the ones that have advanced too
		far. (With difficulty this could be done in a single pass through the list of drops, but it is much
		more cumbersome to do that way.)
	\end{itemize}

	When using function composition the normal template for the function is discarded, and the body
	of the function has two or more function compositions. So in the case of arrange-images the
	function design would look like this:
	
	\begin{minted}{racket}
	;; ListOfImage -> Image
	;; arrange images left to right in increasing order of size
	(check-expect (arrange-images (list I1 I3 I2))
	              (beside I1 I2 I3 BLANK))
	
	(define (arrange-images loi)
		(layout-images (sort-images loi)))
	\end{minted}
	
	Which we read as saying "first sort the images, and then layout the sorted list". At the point this is
	written wish list entries would be created for layout-images and sort-images unless those
	functions already existed.
	
	Tests for a function that uses function composition should be selected to ensure that the function
	is calling all the appropriate functions and composing them properly. For example, assuming that
	images I1, I2 and I3 are in increasing order of size, then this test alone would not be adequate:
	
	\begin{minted}{racket}
	(check-expect (arrange-images (list I1 I2 I3))
	              (beside I1 I2 I3 BLANK))
	\end{minted}
	
	Because a faulty implementation of arrange-images that just calls layout-images would pass.
	Instead a test like the original one above is needed, to ensure that both sort-images and
	layout-images are called. But note that the tests for arrange-images do not themselves need to
	fully test both composed functions. They only need to test the composition. That is why
	arrange-images does not absolutely have to have a base case test. (Although it wouldn't hurt it
	to have one.)
	
	\section{Backtracking Search} \label{sec:back_srch}
	The template for backtracking is:
	
	\begin{minted}{racket}
	(define (fn-for-x x)
		(... (fn-for-lox (x-subs x))))
	
	(define (fn-for-lox lox)
		(cond [(empty? lox) false]
		      [else
		        ;is first child successful?
		        (if (not (false? (fn-for-x (first lox))))
		            ;if so produce that
		            (fn-for-x (first lox))
		            ;or try rest of children
		            (fn-for-lox (rest lox)))]))
	\end{minted}
	
	Note that this template incorporates the template for an n-ary tree, where the tree nodes have
	type x and x-subs produces the children of a tree node given the node. The backtracking works
	as commented above. Once we have local expressions we tend to write the backtracking search
	template as follows:
	
	\begin{minted}{racket}
	(define (backtracking-fn x)
		(local [(define (fn-for-x x)
		          (... (fn-for-lox (x-subs x))))
		        
		        (define (fn-for-lox lox)
		        	(cond [(empty? lox) false]
			              [else
			                (local 
			                  [(define try 
			                    (fn-for-x 
			                      (first lox)))]
			                  (if (not (false? try))
			                    try
			                     (fn-for-lox 
			                       (rest lox))))]))]
		(fn-for-x x)))
	\end{minted}
	\pagebreak
	\section{Generative Recursion} \label{sec:gen_recur}
	The template for generative recursion is:
	
	\begin{minted}{racket}
	(define (genrec-fn d)
		(cond [(trivial? d) (trivial-answer d)]
		      [else
		        (... d
		             (genrec-fn (next-problem d)))]))
	\end{minted}
	
	\chapter{Accumulators} \label{ch:accumulators}
	There are three general ways to use an accumulator in a recursive function (or set of mutually
	recursive functions):
	\begin{enumerate}
		\item To preserve context otherwise lost in structural recursion.
		\item To make a function tail recursive by preserving a representation of the work done so far (aka a
		result-so-far accumulator).
		\item To make a function tail recursive by preserving a representation of the work remaining to do
		(aka a worklist accumulator).
	\end{enumerate}
	
	The same basic recipe covers all three forms of accumulators. The main example shown here is a
	context preserving accumulator.
	
	\section{The basic recipe and context preserving accumulators.}
	\subsection*{Signature, purpose, stub and examples}
	Design of the function begins normally, with signature, purpose, stub and examples.
	
	\begin{minted}{racket}
	;; (listof X) -> (listof X)
	;; produce list formed by keeping 
	;; the 1st, 3rd, 5th and so on elements of lox
	(check-expect (skip1 (list "a" "b" "c" "d")) 
	              (list "a" "c"))
	(check-expect (skip1 (list 0 1 2 3 4)) (list 0 2 4))
	
	(define (skip1 lox) empty) ;stub
	\end{minted}
	
	\subsection*{Templating}
	The template step is a 3 part process. The first step is to template normally according to the rules
	for structural recursion, i.e. template according to the (listof X) parameter.
	
	\begin{minted}{racket}
	(define (skip1 lox)
		(cond [(empty? lox) (...)]
		      [else
		        (... (first lox)
		        (skip1 (rest lox)))]))
	\end{minted}
	
	The next step is to encapsulate that function in an outer function and local. As part of this step
	give the outer function parameter a different name than the inner function parameter. Note that if
	you are working with multiple mutually recursive functions they call get wrapped in a single outer
	function.
	
	\begin{minted}{racket}
	(define (skip1 lox0)
		(local [(define (skip1 lox)
		       (cond [(empty? lox) (...)]
		             [else
		               (... (first lox)
		               (skip1 (rest lox)))]))]
		(skip1 lox0)))
	\end{minted}
	
	Now add the accumulator parameter to the inner function. In addition, add ... or more substantial
	template expressions in each place that calls the inner function. During this step treat the
	accumulator parameter as atomic.
	
	\begin{minted}{racket}
	(define (skip1 lox0)
		(local [(define (skip1 lox acc)
		                (cond [(empty? lox) (... acc)]
		                      [else
		                        (... acc
		                          (first lox)
		                          (skip1 (rest lox)
		                                 (... acc 
		                                   (first lox))))]))]
		(skip1 lox0 ...)))
	\end{minted}
	
	\subsection*{Accumulator type, invariant and examples}
	The next step is to work out what information the accumulator will represent and how it will do
	that. Will the accumulator serve to represent some context that would otherwise be lost to
	structural recursion? Or, to support tail recursion will it represent some form of result so far? Or
	will it represent a work list of some sort. In many cases this is clear before reaching this stage of
	the design process. In other cases examples can be used to work this out. But in all cases
	examples are useful to work out exactly how the accumulator will represent the information.
	
	In this case we need to know, in each recursive call to the inner function, whether the current first
	item in the list should be kept or skipped. There are many ways to represent this, but one simple
	way is to use a natural that represents how far into the original list (lox0) we have traveled. If we
	assume the call to the top-level definition of skip1 is (skip1 (list 0 1 2 3 4)), then the progression of
	calls to the internal skip1 would be as follows:
	
	\begin{minted}{racket}
	;the 0 is the 1st element of lox0 
	;(using 1 based indexing)
	(skip1 (list 0 1 2 3 4) 1)	
	(skip1 (list 1 2 3 4) 2) ;the 1 is the 2nd element
	(skip1 (list   2 3 4) 3) ;the 2 is the 3rd
	(skip1 (list     3 4) 4) ;the 3 is the 4th
	(skip1 (list       4) 5) ;the 4 is the 5th
	(skip1 (list        ) 6)
	\end{minted}
	
	Note that the accumulator value is not constant, but it always represents the position of the
	current (first lox) in the original list lox0. (You can see here why we renamed the parameter to the
	outer function, it makes it easier to describe the relation between the original value lox0 and the
	value in each recursive call lox.)
	
	These examples of the progression of calls to the internal recursive function(s) allow us to work
	out clearly the accumulator type, as well as its invariant, which describes what is constant about
	the accumulator as it changes in other words, what property it always represents. In this case the
	type is Natural, and the invariant is the 1 based index of (first lox) in lox0. So when the
	accumulator is an odd number we will keep (first lox) in the result, when it is even we will skip
	(first lox).
	
	\begin{minted}{racket}
	(define (skip2 lox0)
		;; acc is Natural
		;how many elements of lox to keep before next skip
		;; (skip2 (list 0 1 2 3 4) 1)
		;; (skip2 (list   1 2 3 4) 0)
		;; (skip2 (list     2 3 4) 1)
		;; (skip2 (list       3 4) 0)
		;; (skip2 (list         4) 1)
		;; (skip2 (list          )
		
		(local [(define (skip2 lox acc)
		        (cond [(empty? lox) (... acc)]
		              [else
		                (... acc
		                     (first lox)
		                     (skip2 (rest lox)
		                            (... acc 
		                              (first lox))))]))]
		(skip2 lox0 ...)))
	\end{minted}
	
	\subsection*{Complete the code}
	At this point the function definition can be completed by using the signature, purpose, examples,
	accumulator type, invariant and examples to fill in the details. When doing so, note three distinct
	aspects of coding with the accumulator invariant. Initializing the accumulator happens in the
	trampoline, and involves providing an initial value of the accumulator that satisfies the invariant.
	Exploiting the invariant involves counting on the accumulator to always represent what the
	accumulator describes. Preserving the invariant happens in recursive calls to the function where a
	(possibly) new value is provided for the accumulator argument. Preserving the invariant means
	making sure that the value provided in the recursive call satisfies the invariant.
	
	\begin{minted}{racket}
	(define (skip1 lox0)
		;; acc is Natural
		; 1 based index of (first lox) in lox0
		;; (skip1 (list 0 1 2 3 4) 1) ;0 should be kept
		;; (skip1 (list   1 2 3 4) 2) ;1 should be skipped
		;; (skip1 (list     2 3 4) 3) ;0 should be kept
		;; (skip1 (list       3 4) 4) ;1 should be skipped
		;; (skip1 (list         4) 5) ;0 should be kept
		;; (skip1 (list          ) 6)
		(local [(define (skip1 lox acc)
		        (cond [(empty? lox) empty]
		              [else
		                (if (even? acc)
		                    (skip1 (rest lox) (add1 acc))
		                    (cons (first lox)
		                          (skip1 (rest lox) 
		                            (add1 acc))))]))]
		(skip1 lox0 1)))
	\end{minted}
	
	\section{Tail Recursion}
	To make the function tail recursive, all recursive calls must be in tail position. For functions that
	operate on flat structures (data that has only one reference cycle), this can be accomplished by
	using an accumulator to represent build up information about the final result through the series of
	recursive calls. We often name these accumulators rsf but it is worth noting that some functions
	require more than one result so far accumulator (see average).
	
	Making functions that operate on data with more than one cycle in the graph (such as arbitrary-
	arity trees) usually requires the use of an accumulator to build up the data that still remains to be
	operated on. This is a worklist accumulator, often called todo.
	
	\chapter{Template Blending} \label{ch:temp_blend}
	To understand template blending its important to understand that templates are what we know
	about the core of a function (or set of functions) before we get to the details. Data driven (or
	structural recursion) templates are that, backtracking templates are that, and generative recursion
	templates are that.
	
	In some cases we know more than one thing about the core structure of a function (or set of
	functions). In the sudoku-solver for example three different templates apply to the solve
	functions.
	\begin{description}
		\item[arbitrary-arity tree] - we consider each board to have a set of next boards formed by filling the
		first empty cell with the numbers from 1 - 9. So this forms an arbitrary-arity tree (the arity is
		actually [0, 9]).
		\item[generative recursion] - while each board has a set of next boards that set is included in the
		representation of the board. Instead those next boards have to be generated. So we have a
		generated arbitrary-arity tree.
		\item[backtracking search] - in addition we need to do a backtracking search over the arbitrary-arity
		tree.
	\end{description}

	In template blending we take multiple templates that contribute to the structure of a function (or
	functions) and combine them together.
	
	We have the following for the solve function:
	
	\begin{minted}{racket}
	;; Board -> Board or false
	;; produce a solution for bd; false if bd is unsolvable
	;; Assume: bd is valid
	(check-expect (solve BD4) BD4s)
	(check-expect (solve BD5) BD5s)
	(check-expect (solve BD7) false)
	
	(define (solve bd) false) ;stub
	\end{minted}
	
	Now let's start with the template for arbitrary-arity tree. Remember that the template for an
	arbitrary-arity tree involves a mutual recursion. In this case, we would have a function that
	consumes Board (ie. solve--bd) and calls another function (ie. solve--lobd) that does
	something to the (listof Board) that is supposed to come with the Board (a Board doesn't actually
	come with a (listof Board) but we will deal with it using generative recursion later). In order to
	complete the mutual recursion, we need to call solve--bd inside the function solve--lobd. So
	now we have the following template:
	
	\begin{minted}{racket}
	(define (solve bd)
		(local [(define (solve--bd bd)
		        (... (solve--lobd (bd-subs bd))))
		        
		        (define (solve--lobd lobd)
		                (cond [(empty? lobd) (...)]
		                      [else
		                        (... 
		                          (solve--bd (first lobd))
		                          (solve--lobd 
		                            (rest lobd)))]))]
		(solve--bd bd)))
	\end{minted}
	
	Keep in mind that the bd-subs selector doesn't exist because Board does not keep a (listof
	Board). We need to deal with it using generative recursion.
	
	Let's blend this template with the generative recursion template. The generative recursion
	template looks like this:
	
	\begin{minted}{racket}
	(define (genrec-fn d)
		(if (trivial? d)
		    (trivial-answer d)
		    (... d
		        (genrec-fn (next-problem d)))))
	\end{minted}
	
	Since we know that solve--bd must generate a (listof Board) to pass to solve--lobd, we must
	blend the template into the solve--bd function. Then we would have:
	
	\begin{minted}{racket}
	(define (solve bd)
		(local [(define (solve--bd bd)
		        (if (solved? bd)
		            bd
		            (solve--lobd (next-boards bd))))
		            
		        (define (solve--lobd lobd)
		                (cond [(empty? lobd) (...)]
		                      [else
		                        (... 
		                          (solve--bd (first lobd))
		                          (solve--lobd 
		                            (rest lobd)))]))]
		(solve--bd bd)))
	\end{minted}
	
	Notice the following:
	\begin{itemize}
		\item bd-subs is changed to next-boards because it suggests that we are generating new boards.
		\item trivial? from the generative recursion template is changed to solved? because the trivial
		case (ie. the case where the recursion should stop at) is when the board is solved.
		\item trivial-answer from the generative recursion template is omitted because in this case, the
		trivial answer is the board that is solved, which is represented by bd.
	\end{itemize}

	Lastly, we need to blend in the template for backtracking search. The template is:
	
	\begin{minted}{racket}
	(define (backtracking-fn x)
		(local [(define (fn-for-x x)
		          (... (fn-for-lox (x-subs x))))
		        (define (fn-for-lox lox)
		                (cond [(empty? lox) false]
		                      [else
		                        (local [(define try 
		                          (fn-for-x (first lox)))]
		                        (if (not (false? try))
		                          try
		                          (fn-for-lox 
		                            (rest lox))))]))]
		(fn-for-x x)))
	\end{minted}
	
	We need to blend the template for fn-for-lox from above into the function solve-lobd.
	
	\begin{minted}{racket}
	(define (solve bd)
		(local [(define (solve--bd bd)
		          (if (solved? bd)
		              bd
		              (solve--lobd (next-boards bd))))
		              
		  (define (solve--lobd lobd)
		        (cond [(empty? lobd) false]
		              [else
		                (local 
		                  [(define try 
		                    (solve--bd (first lobd)))]
		                  (if (not (false? try))
		                      try
		                      (solve--lobd 
		                        (rest lobd))))]))]
		(solve--bd bd)))
	\end{minted}
	
	The template allows the solve--lobd function to do the following:
	\begin{itemize}
		\item If we have reached the end of the (listof Board) (ie. if lobd is empty - the base case), then
		produce false, meaning that the (listof Board) has no solution.
		\item If the (listof Board) is not empty, then try to solve the first board in the list (ie. (solve--bd
		(first lobd))).
		\item If the outcome of the try is not false (ie. (not (false? try))), then produce that outcome
		because it means the board has been solved.
		\item If the outcome is false, then recurse and try the rest of the list (ie. (solve--lobd (rest
		lobd))).
	\end{itemize}
	
	Now we have the complete template below.
	
	\begin{minted}{racket}
	(define (solve bd)
		(local [(define (solve--bd bd)
		        (if (solved? bd)
		            bd
		            (solve--lobd (next-boards bd))))
		            
		        (define (solve--lobd lobd)
		                (cond [(empty? lobd) false]
		                      [else
		                        (local [(define try 
		                          (solve--bd 
		                            (first lobd)))]
		                          (if (not (false? try))
		                            try
		                            (solve--lobd 
		                              (rest lobd))))]))]
		(solve--bd bd)))
	\end{minted}
	
	The next step to completing the function is to implement the two helper functions - solved? and
	next-boards.
	
	\chapter{Abstraction}
	\section{Abstraction From Examples} \label{sec:abs_from_eg}
	We design abstract functions in the opposite order of the normal HtDF recipe. We always want to
	do the easiest thing first, and with the abstract function design processes getting the working
	function definition is the easiest thing to do. In fact going through the recipe in the opposite order
	exactly goes from easiest to hardest.
	\begin{enumerate}
		\item Identify two or more fragments of highly repetitive code. In general these can be expressions
		that appear within functions, or they can be entire functions. The rest of this recipe is tailored to
		the case where entire functions have been chosen.
		\item Arrange the two functions so that it is easy to see them at the same time.
		\item Identify one or more points where the functions differ (points of variance). Do not count
		differences in function names or parameter names as points of variance.
		\item Copy one function definition to make new one
		\begin{itemize}
			\item give the new function a more general name
			\item add a new parameter for each point of variance
			\item update any recursive calls to use new name and add new parameters in recursive calls
			\item use the appropriate new parameter at each point of variability
			\item rename other parameters to be more abstract (lon to lox for example)
		\end{itemize}
		\item Adapt tests from original functions to new abstract function
		\begin{itemize}
			\item be sure to test variability
			\item attempt to test behavior of the abstract function beyond that exercised by the two examples
		\end{itemize}
		\item Develop an appropriately abstract purpose based on the examples.
		\item Develop an appropriate signature for the abstract function; in many cases the signature will
		include type parameters.
		\item Rewrite the body of the two original functions to call the abstract function.
	\end{enumerate}

	\section{Abstraction From Type Comments} \label{sec:abs_from_type_comm}
	We design abstract functions in the opposite order of the normal HtDF recipe. We always want to
	do the easiest thing first, and with the abstract function design processes getting the working
	function definition is the easiest thing to do. In fact going through the recipe in the opposite order
	exactly goes from easiest to hardest.
	\begin{enumerate}
		\item if there are templates for mutually recursive functions first encapsulate them in a single
		template with local
		\item replace each ... in the templates with a new parameter; for (...) remove the parens
		\item develop examples (check-expects)
		\item develop abstract purpose from examples
		\item develop abstract signature from concrete examples
	\end{enumerate}

	Let's now go through generating a fold function for (listof X). Here is the type comment for
	(listof X):
	
	\begin{minted}{racket}
	;; ListOfX is one of:
	;; - empty
	;; - (cons X ListOfX)
	\end{minted}
	
	We can generate the following template based on this type comment:
	
	\begin{minted}{racket}
	(define (fn-for-lox lox)
		(cond [(empty? lox) (...)]
		      [else
		        (... (first lox)
		             (fn-for-lox (rest lox)))]))
	\end{minted}
	
	Since there is no mutual recursion in the template, we can skip step 1. According to step 2, we
	need to replace each $\ldots$ with a new parameter. We also need to rename the function to fold.
	Here is what we would have after step 2:
	
	\begin{minted}{racket}
	(define (fold fn b lox)
		(cond [(empty? lox) b]
		      [else
		        (fn (first lox)
		            (fold fn b (rest lox)))]))
	\end{minted}
	
	Note that we called the $\ldots$ in place of the base case is named to b, and the $\ldots$ in place of the
	function is named to fn.
	
	For step 3, we should write some examples to test for the function and also find out what we can
	do with the fold function:
	
	\begin{minted}{racket}
	; sum of the numbers in the list
	(check-expect (fold + 0 (list 1 2 3)) 6)
	
	; product of the numbers in the list
	(check-expect (fold * 1 (list 1 2 3)) 6)
	
	; append the strings in the list
	(check-expect (fold string-append "" 
	                     (list "a" "bc" "def")) "abcdef")
	
	; sum of the areas of the images in the list
	(check-expect (local [(define (total-area i a)
		             (+ (* (image-width i)
		                   (image-height i))
		               a))]
			(fold total-area 0 (list 
		                     (rectangle 20 40 "solid" "red")
		                     (right-triangle 10 20 "solid" "red")
		                     (circle 20 "solid" "red"))))
		(+ (* 20 40) (* 10 20) (* 40 40)))
	\end{minted}
	
	Step 4 requires us to develop an abstract purpose from the examples. In this example, we can just
	write:
	
	\begin{minted}{racket}
	;; the abstract fold function for (listof X)
	\end{minted}
	
	In the last step, we need to determine the abstract signature from the examples that we have
	written. The more diverse the examples that we have, the easier it is to come up with the correct
	signature.
	
	From the function defintion for fold, we can first reason the following for the signature of fold:
	\begin{itemize}
		\item the parameter lox is of the type (listof X)
		\item the parameter b is of the same type as what the fold function produces because it is the base
		case
		\item the predicate function fn takes in two arguments
		\item the predicate function fn produces the same type as what the fold function produces
		\item the first parameter for the predicate function fn is of the type X
		\item the second parameter for the predicate function fn is of the same type as what the fold
		function produces
	\end{itemize}

	From the above reasonings, we can narrow the signature down to:
	\begin{minted}{racket}
	;; (X ??? -> ???) ??? (listof X) -> ???
	\end{minted}
	
	Note that all of the ??? are of the same type according to the above reasoning. Now the question
	is whether ??? has to be the same type as X or can be something else (ie. Y).
	From the fourth example in step 3, we can see that the signature for the predicate function
	total-area is:
	
	\begin{minted}{racket}
	;; Image Number -> Number
	\end{minted}
	
	In this case, the X is Image and the ??? is Number. This is an example of ??? being something
	different than X. Therefore, we can conclude that the abstract signature for fold is:
	
	\begin{minted}{racket}
	;; (X Y -> Y) Y (listof X) -> Y
	\end{minted}
	
	\section{Using Abstract Functions} \label{sec:using_abs_func}
	The template for using a built-in abstract function like filter is:
	\begin{minted}{racket}
	;; (listof Number) -> (listof Number)
	
	;; produce only positive? elements of lon
	
	;; tests elided
	
	;; template as call to abstract function
	(define (only-positive lon)
		(filter ... lon))
	\end{minted}
	
	Now we note that the type of lon is (listof Number); and the signature of filter is (X -> Boolean)
	(listof X) -> (listof X). This means that that signature of the function passed to filter is
	(Number -> Boolean) so we can further decorate the template as follows:
	
	\begin{minted}{racket}
	(define (only-positive lon)
		;(Number -> Boolean)
		(filter ... lon))
	\end{minted}
	
	as a note to ourselves about the signature of the function we replace $\ldots$ with.
\end{document}
